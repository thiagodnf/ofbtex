\section{Introdução}
\label{sec:introducao}

Em agosto de 2016, o Programa Estadual de Concessões e Parcerias Público-Privadas do Estado do Ceará foi lançado oficialmente pelo Governador do Estado, elencando um conjunto de projetos prioritários de grande vulto com impactos econômicos e sociais relevantes para o Estado que, a partir de estudos preliminares, apresentaram potencial para desenvolver alianças com o setor privado. Desde então, novos projetos foram incluídos na carteira de prioritários e iniciou-se a estruturação de alguns selecionados por meio de Procedimentos de Manifestação de Interesse e, internamente, por meio de grupos de trabalho, havendo um padrão de avaliação que exige aderência a legislação atual e aos aspectos teóricos que uma PPP carreia.  

A Lei Estadual n° 14.391/2009, observadas as normas gerais previstas na legislação federal dada pela Lei Federal nº 11.079, vedam os contratos de PPP cujo valor for inferior a 20 milhões de reais, aqueles onde o período de prestação do serviço seja inferior a cinco anos e os contratos que tenham como objeto único o fornecimento de mão-de-obra, instalação de equipamentos ou a execução de obra pública. A mesma lei estadual mantem o Conselho Gestor de Parcerias Público-Privadas – CGPPP com competências delimitadas pela Lei Estadual nº 13.557/04, com alguns acréscimos, cabe ao conselho aprovar a execução dos projetos de PPP, disciplinar os procedimentos para celebração dos contratos, autorizar a abertura de licitação e aprovar o seu edital, apreciar os relatórios de execução e deliberar sobre os casos omissos. Conforme o Decreto nº 29.801/2009 em seu art. 1º o CGPPP é composto pelo Secretário de Estado do Planejamento e Gestão, Secretário de Estado da Fazenda, Secretário de Estado da Casa Civil, Procurador-Geral do Estado e um Secretário de Estado da Infraestrutura cuja área de competência seja pertinente ao objeto da PPP. A CGPPP tem, dentre suas competências, analisar os projetos, estudos, levantamentos ou investigações que forem elaborados por pessoas físicas ou jurídicas não pertencentes à Administração Pública direta ou indireta. Desta linha, propõe-se aqui com base nos Fatores Críticos de Sucesso – FCS apontados em \citeonline{osei2015review} e Sehgal e Dubey (2019), ajustes a carta convite e um procedimento sequencial para avaliar projetos apresentados por pessoas físicas ou jurídicas não pertencentes à Administração Pública direta ou indireta de forma a dar celeridade, imparcialidade, cientificidade e precisão nas análises da CGPPP.

A proposta de modelo de análise segue o modelo do Banco Mundial Tool kit, com adaptações a realidade local, particularmente no que tange ao equilíbrio fiscal do estado e a subordinação aos relatórios fiscais ao Tesouro Nacional, pois se identificou na proposta do Banco Mundial características para a avaliação da CGPPP, contudo, há de se aventar avaliações qualitativas aos moldes de propostas como as de Five Case Model e do projeto BENEFIT. De forma teórica, adotou-se os insights pertencentes no modelo de \citeonline{engel2008public} por aderência a conceitos internacionais e ser embasado em projetos internacionais.

Considerações a cerca dos contratos, renegociação e modelo do leilão exigem estudos aprofundados do modelo de \citeonline{engel2008public} com complementos da teoria dos contratos incompletos, desenvolvida por Grossman and Hart (1986), \citeonline{hart1990property}, \citeonline{hart1995firms} a distribuição de riscos pela teoria de incentivos em \citeonline{tirole1999incomplete} o que deve ficar para um produto quatro com a renovação do estudo.


